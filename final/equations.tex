\documentclass{article}

\usepackage{amsmath, amsthm, amssymb, amsfonts}
\usepackage{geometry}
\usepackage[utf8]{inputenc}
\usepackage{import}
\usepackage{pdfpages}
\usepackage{transparent}
\usepackage{xcolor}
\usepackage{relsize}

\newcommand{\Z}{\mathbb{Z}}
\newcommand{\R}{\mathbb{R}}
\newcommand{\Q}{\mathbb{Q}}
\newcommand{\N}{\mathbb{N}}
\newcommand{\F}{\mathbb{F}}
\newcommand{\C}{\mathbb{C}}

\newcommand{\incfig}[2][1]{%
    \def\svgwidth{#1\columnwidth}
    \import{./figures/}{#2.pdf_tex}
}

\pdfsuppresswarningpagegroup=1\usepackage[english]{babel}

\begin{document}

\title{Equations to Model Decomposition Rate}
\author{\textbf{Ansel Rausis} \ }
\date{2023-10-23}

\noindent
\section*{Environment}
$ D: $ Woody Decomposition Rate\\
$ K: $ Carrying Capacity\\
$ \beta : $  Effect of environment on growth\\ 
$ H: $ Environmental severity\\
$ N(t): $ Total population \\
$ S(t): $ Population of Sapotrophic Fungi\\
$ j:  $ Number of fungal species

\section*{Fungus}
$ r_i: $ Hyphal Extension Rate\\
$ h_i: $ Ideal severity\\
$ n_i(t): $ Population\\
$ m_i: $ Moisture tolerance \\
$ s_i: $ Sapotrophic/not sapotrophic\\ 
$ c_i: $ Competitiveness

\section*{The equations}

\begin{equation} \label{Population}
    N(t) = \mathlarger{\mathlarger{\sum}}_{i=0}^{j} n_i
\end{equation}

\begin{equation} \label{Sapotrophic}
    S(t) = \mathlarger{\mathlarger{\sum}}_{i=0}^{j} 
    \begin{cases}
	    0 & s_i = 0\\
	    n_i & s_i = 1
    \end{cases}
\end{equation}
\noindent
Equations \ref{Population} and \ref{Sapotrophic} define our population and sapotrophic populations using sigma notation.

\begin{equation}\label{Growth rate}
    \frac{dn_i}{dt} = r_i \left(c_i - \frac{\left| h_i - H\right|}{m_i} - \frac{N}{K} \right)
\end{equation}

\noindent
Equation \ref{Growth rate} shows how we have chosen to simulate fungal performance. Note that a single fungus under ideal conditions would grow to the carrying capacity ($ K $ ) at ti's growth rate ($ r_i $ ). If the fungus were outside of it's ideal environment, then this would be indiciated in a difference between $ h_i $ and $ H $, which would lower the growth rate as well as the maximum capacity that the fungus could reach. This effect would be mitigated by a fungus having a higher moisture tolerance ($ m_i $ ). A similar effect occurs based on the ($ c_i $ ) of the fungus, which represents how well the fungus competes with other fungi in it's environment. A lower $ c_i $ would result in slower growth rate and a lower capacity for the fungus. This effect would be amplified as there is more biomass of competing fungi in the environment. 


 \begin{equation}
     D(t) \propto S(t)
 \end{equation}
 \noindent
 This one is still up for some alterations, but I currently have our decomposition rate being proportional to the population of sapotrophic fungi.
 
 \end{document}
